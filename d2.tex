%%%%%%%%%%%%  Generated using docx2latex.com  %%%%%%%%%%%%%%

%%%%%%%%%%%%  v2.0.0-beta  %%%%%%%%%%%%%%

\documentclass[12pt]{article}
\usepackage{amsmath}
\usepackage{latexsym}
\usepackage{amsfonts}
\usepackage[normalem]{ulem}
\usepackage{array}
\usepackage{amssymb}
\usepackage{graphicx}
\usepackage[backend=biber,
style=numeric,
sorting=none,
isbn=false,
doi=false,
url=false,
]{biblatex}\addbibresource{bibliography.bib}

\usepackage{subfig}
\usepackage{wrapfig}
\usepackage{wasysym}
\usepackage{enumitem}
\usepackage{adjustbox}
\usepackage{ragged2e}
\usepackage[svgnames,table]{xcolor}
\usepackage{tikz}
\usepackage{longtable}
\usepackage{changepage}
\usepackage{setspace}
\usepackage{hhline}
\usepackage{multicol}
\usepackage{tabto}
\usepackage{float}
\usepackage{multirow}
\usepackage{makecell}
\usepackage{fancyhdr}
\usepackage[toc,page]{appendix}
\usepackage[hidelinks]{hyperref}
\usetikzlibrary{shapes.symbols,shapes.geometric,shadows,arrows.meta}
\tikzset{>={Latex[width=1.5mm,length=2mm]}}
\usepackage{flowchart}\usepackage[paperheight=11.0in,paperwidth=8.5in,left=1.73in,right=1.73in,top=1.5in,bottom=1.5in,headheight=1in]{geometry}
\usepackage[utf8]{inputenc}
\usepackage[T1]{fontenc}
\TabPositions{0.5in,1.0in,1.5in,2.0in,2.5in,3.0in,3.5in,4.0in,4.5in,5.0in,}

\urlstyle{same}


 %%%%%%%%%%%%  Set Depths for Sections  %%%%%%%%%%%%%%

% 1) Section
% 1.1) SubSection
% 1.1.1) SubSubSection
% 1.1.1.1) Paragraph
% 1.1.1.1.1) Subparagraph


\setcounter{tocdepth}{5}
\setcounter{secnumdepth}{5}


 %%%%%%%%%%%%  Set Depths for Nested Lists created by \begin{enumerate}  %%%%%%%%%%%%%%


\setlistdepth{9}
\renewlist{enumerate}{enumerate}{9}
		\setlist[enumerate,1]{label=\arabic*)}
		\setlist[enumerate,2]{label=\alph*)}
		\setlist[enumerate,3]{label=(\roman*)}
		\setlist[enumerate,4]{label=(\arabic*)}
		\setlist[enumerate,5]{label=(\Alph*)}
		\setlist[enumerate,6]{label=(\Roman*)}
		\setlist[enumerate,7]{label=\arabic*}
		\setlist[enumerate,8]{label=\alph*}
		\setlist[enumerate,9]{label=\roman*}

\renewlist{itemize}{itemize}{9}
		\setlist[itemize]{label=$\cdot$}
		\setlist[itemize,1]{label=\textbullet}
		\setlist[itemize,2]{label=$\circ$}
		\setlist[itemize,3]{label=$\ast$}
		\setlist[itemize,4]{label=$\dagger$}
		\setlist[itemize,5]{label=$\triangleright$}
		\setlist[itemize,6]{label=$\bigstar$}
		\setlist[itemize,7]{label=$\blacklozenge$}
		\setlist[itemize,8]{label=$\prime$}

\setlength{\topsep}{0pt}\setlength{\parskip}{8.04pt}
\setlength{\parindent}{0pt}

 %%%%%%%%%%%%  This sets linespacing (verticle gap between Lines) Default=1 %%%%%%%%%%%%%%


\renewcommand{\arraystretch}{1.3}


%%%%%%%%%%%%%%%%%%%% Document code starts here %%%%%%%%%%%%%%%%%%%%



\begin{document}

\vspace{\baselineskip}
\begin{justify}
{\fontsize{14pt}{16.8pt}\selectfont \textbf{Problem 6. [60 Marks]}\par}
\end{justify}\par



%%%%%%%%%%%%%%%%%%%% Table No: 1 starts here %%%%%%%%%%%%%%%%%%%%


\begin{table}[H]
 			\centering
\begin{tabular}{p{2.27in}p{2.27in}}
\hline
%row no:1
\multicolumn{1}{|p{2.27in}}{\textbf{Identifier}} & 
\multicolumn{1}{|p{2.27in}|}{EternityNumbers\_01} \\
\hhline{--}
%row no:2
\multicolumn{1}{|p{2.27in}}{\textbf{Statement}} & 
\multicolumn{1}{|p{2.27in}|}{\cellcolor[HTML]{FFFFFF}As a user, I want my calculator to display the value of $ \pi $  till 15 decimal places } \\
\hhline{--}
%row no:3
\multicolumn{1}{|p{2.27in}}{\textbf{Constraint}} & 
\multicolumn{1}{|p{2.27in}|}{- } \\
\hhline{--}
%row no:4
\multicolumn{1}{|p{2.27in}}{\textbf{Acceptance Criteria}} & 
\multicolumn{1}{|p{2.27in}|}{\cellcolor[HTML]{FFFFFF}Given that I have to determine the value of $ \pi $ . \par \begin{itemize}
	\item I will press the button for $ \pi $ .
\end{itemize} \par \begin{itemize}
	\item The value for $ \pi $  will be displayed till 15 decimal places i.e. 3.1415$ \ldots $ $ \ldots $ .
\end{itemize}} \\
\hhline{--}
%row no:5
\multicolumn{1}{|p{2.27in}}{\textbf{Priority}} & 
\multicolumn{1}{|p{2.27in}|}{Should Have} \\
\hhline{--}
%row no:6
\multicolumn{1}{|p{2.27in}}{\textbf{Estimate}} & 
\multicolumn{1}{|p{2.27in}|}{½ Hour} \\
\hhline{--}

\end{tabular}
 \end{table}


%%%%%%%%%%%%%%%%%%%% Table No: 1 ends here %%%%%%%%%%%%%%%%%%%%


\vspace{\baselineskip}


%%%%%%%%%%%%%%%%%%%% Table No: 2 starts here %%%%%%%%%%%%%%%%%%%%


\begin{table}[H]
 			\centering
\begin{tabular}{p{2.28in}p{2.28in}}
\hline
%row no:1
\multicolumn{1}{|p{2.28in}}{\textbf{Identifier}} & 
\multicolumn{1}{|p{2.28in}|}{EternityNumbers\_02} \\
\hhline{--}
%row no:2
\multicolumn{1}{|p{2.28in}}{\textbf{Statement}} & 
\multicolumn{1}{|p{2.28in}|}{\cellcolor[HTML]{FFFFFF}As a user, I should be able to enter the value of $ \pi $  to calculate the circumference of circle in the calculator.} \\
\hhline{--}
%row no:3
\multicolumn{1}{|p{2.28in}}{\textbf{Constraint}} & 
\multicolumn{1}{|p{2.28in}|}{Number entered by the user for the radius should not be negative. } \\
\hhline{--}
%row no:4
\multicolumn{1}{|p{2.28in}}{\textbf{Acceptance Criteria}} & 
\multicolumn{1}{|p{2.28in}|}{\cellcolor[HTML]{FFFFFF}Given that I have to calculate the circumference of a circle with radius 3.  \par \begin{itemize}
	\item I will enter the digit 2 and then press multiplication operator. \par 	\item When I will press $ \pi $ , the calculator will display its value i.e. 3.141$ \ldots $ $ \ldots $  \par 	\item After pressing the multiplication operator again, I will press 3 and then =. This will give me the result for the circumference.
\end{itemize}} \\
\hhline{--}
%row no:5
\multicolumn{1}{|p{2.28in}}{\textbf{Priority}} & 
\multicolumn{1}{|p{2.28in}|}{Must Have} \\
\hhline{--}
%row no:6
\multicolumn{1}{|p{2.28in}}{\textbf{Estimate}} & 
\multicolumn{1}{|p{2.28in}|}{½ Hour} \\
\hhline{--}

\end{tabular}
 \end{table}


%%%%%%%%%%%%%%%%%%%% Table No: 2 ends here %%%%%%%%%%%%%%%%%%%%


\vspace{\baselineskip}

\vspace{\baselineskip}


%%%%%%%%%%%%%%%%%%%% Table No: 3 starts here %%%%%%%%%%%%%%%%%%%%


\begin{table}[H]
 			\centering
\begin{tabular}{p{2.27in}p{2.27in}}
\hline
%row no:1
\multicolumn{1}{|p{2.27in}}{\textbf{Identifier}} & 
\multicolumn{1}{|p{2.27in}|}{EternityNumbers\_03} \\
\hhline{--}
%row no:2
\multicolumn{1}{|p{2.27in}}{\textbf{Statement}} & 
\multicolumn{1}{|p{2.27in}|}{\cellcolor[HTML]{FFFFFF}As a user, I want my calculator to allow me to choose the precision of \textcolor[HTML]{222222}{$ \pi $  }so that I can use it as per my requirement.} \\
\hhline{--}
%row no:3
\multicolumn{1}{|p{2.27in}}{\textbf{Constraint}} & 
\multicolumn{1}{|p{2.27in}|}{-} \\
\hhline{--}
%row no:4
\multicolumn{1}{|p{2.27in}}{\textbf{Acceptance Criteria}} & 
\multicolumn{1}{|p{2.27in}|}{\cellcolor[HTML]{FFFFFF}Given that I have to calculate the area of circle and this will use $ \pi $ . I should be able to choose the value of $ \pi $  up to 2 decimal places as it will give me the required result \par \begin{itemize}
	\item While using the formula, I will press $ \pi $ . \par 	\item I will press $``$C$"$  to choose its precision. \par 	\item  I will choose 2 and therefore will get the value 3.14 that I can use to calculate the area.
\end{itemize}} \\
\hhline{--}
%row no:5
\multicolumn{1}{|p{2.27in}}{\textbf{Priority}} & 
\multicolumn{1}{|p{2.27in}|}{Must Have} \\
\hhline{--}
%row no:6
\multicolumn{1}{|p{2.27in}}{\textbf{Estimate}} & 
\multicolumn{1}{|p{2.27in}|}{½ Hour} \\
\hhline{--}

\end{tabular}
 \end{table}


%%%%%%%%%%%%%%%%%%%% Table No: 3 ends here %%%%%%%%%%%%%%%%%%%%


\vspace{\baselineskip}


%%%%%%%%%%%%%%%%%%%% Table No: 4 starts here %%%%%%%%%%%%%%%%%%%%


\begin{table}[H]
 			\centering
\begin{tabular}{p{2.28in}p{2.28in}}
\hline
%row no:1
\multicolumn{1}{|p{2.28in}}{\textbf{Identifier}} & 
\multicolumn{1}{|p{2.28in}|}{EternityNumbers\_04} \\
\hhline{--}
%row no:2
\multicolumn{1}{|p{2.28in}}{\textbf{Statement}} & 
\multicolumn{1}{|p{2.28in}|}{As a user, if I enter 2 or more operators consecutively, the operation performed should be of the last operator to enhance usability.} \\
\hhline{--}
%row no:3
\multicolumn{1}{|p{2.28in}}{\textbf{Constraint}} & 
\multicolumn{1}{|p{2.28in}|}{There must be at least 2 operators occurring consecutively in the expression.} \\
\hhline{--}
%row no:4
\multicolumn{1}{|p{2.28in}}{\textbf{Acceptance Criteria}} & 
\multicolumn{1}{|p{2.28in}|}{Given that I have entered 3-+3. The result should be calculated as 3+3 i.e. 6.} \\
\hhline{--}
%row no:5
\multicolumn{1}{|p{2.28in}}{\textbf{Priority}} & 
\multicolumn{1}{|p{2.28in}|}{Must Have} \\
\hhline{--}
%row no:6
\multicolumn{1}{|p{2.28in}}{\textbf{Estimate}} & 
\multicolumn{1}{|p{2.28in}|}{20 minutes} \\
\hhline{--}

\end{tabular}
 \end{table}


%%%%%%%%%%%%%%%%%%%% Table No: 4 ends here %%%%%%%%%%%%%%%%%%%%


\vspace{\baselineskip}


%%%%%%%%%%%%%%%%%%%% Table No: 5 starts here %%%%%%%%%%%%%%%%%%%%


\begin{table}[H]
 			\centering
\begin{tabular}{p{2.28in}p{2.28in}}
\hline
%row no:1
\multicolumn{1}{|p{2.28in}}{\textbf{Identifier}} & 
\multicolumn{1}{|p{2.28in}|}{EternityNumbers\_05} \\
\hhline{--}
%row no:2
\multicolumn{1}{|p{2.28in}}{\textbf{Statement}} & 
\multicolumn{1}{|p{2.28in}|}{My calculator should allow edition on the operator or operand of the current computation until the result is displayed.} \\
\hhline{--}
%row no:3
\multicolumn{1}{|p{2.28in}}{\textbf{Constraint}} & 
\multicolumn{1}{|p{2.28in}|}{At least one operator or operand must be entered.} \\
\hhline{--}
%row no:4
\multicolumn{1}{|p{2.28in}}{\textbf{Acceptance Criteria}} & 
\multicolumn{1}{|p{2.28in}|}{Given that I have to perform the following calculation $``$3+4$\ast$ 2$"$ .  \par \begin{itemize}
	\item I will enter 3+4$\ast$ 2.
\end{itemize} \par \begin{itemize}
	\item Before pressing =, if I want to replace 4 with 5, I can do it.
\end{itemize}} \\
\hhline{--}
%row no:5
\multicolumn{1}{|p{2.28in}}{\textbf{Priority}} & 
\multicolumn{1}{|p{2.28in}|}{Must Have} \\
\hhline{--}
%row no:6
\multicolumn{1}{|p{2.28in}}{\textbf{Estimate}} & 
\multicolumn{1}{|p{2.28in}|}{½ Hour } \\
\hhline{--}

\end{tabular}
 \end{table}


%%%%%%%%%%%%%%%%%%%% Table No: 5 ends here %%%%%%%%%%%%%%%%%%%%


\vspace{\baselineskip}


%%%%%%%%%%%%%%%%%%%% Table No: 6 starts here %%%%%%%%%%%%%%%%%%%%


\begin{table}[H]
 			\centering
\begin{tabular}{p{2.28in}p{2.28in}}
\hline
%row no:1
\multicolumn{1}{|p{2.28in}}{\textbf{Identifier}} & 
\multicolumn{1}{|p{2.28in}|}{EternityNumbers\_06} \\
\hhline{--}
%row no:2
\multicolumn{1}{|p{2.28in}}{\textbf{Statement}} & 
\multicolumn{1}{|p{2.28in}|}{As a user, I want my calculator to keep a record of all the computations till the calculator is turned off.} \\
\hhline{--}
%row no:3
\multicolumn{1}{|p{2.28in}}{\textbf{Constraint}} & 
\multicolumn{1}{|p{2.28in}|}{There must be at least one result for the record to exist.} \\
\hhline{--}
%row no:4
\multicolumn{1}{|p{2.28in}}{\textbf{Acceptance Criteria}} & 
\multicolumn{1}{|p{2.28in}|}{\begin{itemize}
	\item To open the record, press $``$R$"$ . \par 	\item Record empty \par 	\item 3+4=7(stored in record) \par 5+5=10(stored in record) \par 4-2=2(stored in record) \par 	\item Press $``$R$"$ . \par 	\item 7,10,2 \par 	\item Calculator is turned off. \par 	\item After turning on the calculator, press $``$R$"$ . \par 	\item Record empty.
\end{itemize}} \\
\hhline{--}
%row no:5
\multicolumn{1}{|p{2.28in}}{\textbf{Priority}} & 
\multicolumn{1}{|p{2.28in}|}{Should Have} \\
\hhline{--}
%row no:6
\multicolumn{1}{|p{2.28in}}{\textbf{Estimate}} & 
\multicolumn{1}{|p{2.28in}|}{½ Hour } \\
\hhline{--}

\end{tabular}
 \end{table}


%%%%%%%%%%%%%%%%%%%% Table No: 6 ends here %%%%%%%%%%%%%%%%%%%%


\vspace{\baselineskip}

\vspace{\baselineskip}

\vspace{\baselineskip}

\printbibliography
\end{document}